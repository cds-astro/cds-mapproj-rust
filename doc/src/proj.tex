\documentclass[a4paper,twoside,11pt]{article}

% VIGNETTES
\usepackage[pdftex]{thumbpdf} % Vignettes
% NAVIGATION
\usepackage[pdftex,                %
    bookmarks         = true,%     % Signets
    bookmarksnumbered = true,%     % Signets numérotés
    pdfpagemode       = None,%     % Signets/vignettes fermé à l'ouverture
    pdfstartview      = FitH,%     % La page prend toute la largeur
    pdfpagelayout     = SinglePage,% Vue par page
    colorlinks        = true,%     % Liens en couleur
    urlcolor          = magenta,%  % Couleur des liens externes
    pdfborder         = {0 0 0}%   % Style de bordure : ici, pas de bordure
    ]{hyperref}%                   % Utilisation de HyperTeX
\usepackage{url}
\let\urlorig\url         %[fxp] This way \urlorig point to the origianl \url command 
\renewcommand{\url}[1]{  %[fxp] In case we are in French, we switch in english to avoid compatibility errors (because of avtive : in french) with urls
  \begin{otherlanguage}{english}\urlorig{#1}\end{otherlanguage}
}

% MATH ET FORMULES
\usepackage{amsmath} % pour text{} dans un environement math, ...
\usepackage{stmaryrd}    % pour [[  et ]]
\usepackage{txfonts} %pour \mathbb
\usepackage{gensymb} %Pour \degree
\usepackage{yhmath}  %Pour \wideparen
\usepackage{amssymb} % A lot a symbols

%DIVERS
\usepackage{verbatim} % Pour pouvoir utiliser l'environement verbatim

% BIBLIOGRAPHIE BIBTEX
\usepackage{natbib}
\bibliographystyle{sty/bibtex/aa}

\newcommand\araa{ARA\&A}%
\newcommand\aaps{A\&AS}%
\newcommand\aj{AJ}
\newcommand\apj{ApJ}
\newcommand\apjs{{ApJS}}
\newcommand\skytel{{S\&T}}
\newcommand{\aap}{A\&A}

\title{Projections}
\author{F.-X. Pineau, CDS Strasbourg}
\date{\today} 
 

\DeclareMathOperator{\sinc}{sinc}
\DeclareMathOperator{\arcsinc}{arcsinc}

\begin{document}

\maketitle

\tableofcontents \clearpage

\section{Trigonometry formulary}

  \subsection{Fundamental relations}

    \begin{eqnarray}
      \cos^2a + \sin^2x & = & 1 \label{eq:cos2sin2eq1} \\
      \frac{1}{\cos^2}  & = & 1 + \tan^2 x \label{eq:1ocxeq1ptx}
    \end{eqnarray}

  \subsection{Addition formulae}

  \begin{eqnarray}
    \cos(a + b) & = & \cos a \cos b - \sin a \sin b \label{eq:cosapb} \\
    \cos(a - b) & = & \cos a \cos b + \sin a \sin b \label{eq:cosamb} \\
    \sin(a + b) & = & \sin a \cos b + \cos a \sin b \label{eq:sinapb} \\
    \sin(a - b) & = & \sin a \cos b - \cos a \sin b \label{eq:sinamb}
  \end{eqnarray}

  \subsection{Double angle formulae}

  We deduce from the above equations (and Eq. \ref{eq:1ocxeq1ptx}) that:
  \begin{eqnarray}
    \cos(2x) & = & \cos^2 x - \sin^2 x = 2\cos^2 x - 1 = 1 - 2\sin^2 x \label{eq:cos2x} \\
    \sin(2x) & = & 2\sin x \cos x \label{eq:sin2x} \\
    \tan(2x) & = & \frac{2\tan x}{1 + \tan^2x} \label{eq:tan2x}
  \end{eqnarray}

  \subsection{Semi-angle formulae}

    \begin{eqnarray}
      \cos^2 x & = & \frac{1 + \cos(2x)}{2} \label{eq:cosx2} \\
      \sin^2 x & = & \frac{1 - \cos(2x)}{2} \label{eq:sinx2} \\
      \tan^2 x & = & \frac{\sin(2x)}{1 + \cos(2x)} = \frac{1 - \cos(2x)}{\sin(2x)} \label{eq:tanx2}
    \end{eqnarray}


  \subsection{Trigonometric functions depending on inverse trigonometric functions}  

  \begin{eqnarray}
      \cos(\arccos x) &= & x                        \label{eq:cosarccos} \\
      \cos(\arcsin x) &= & \sqrt{1 - x^2}           \label{eq:cosarcsin} \\
      \cos(\arctan x) &= & \frac{1}{\sqrt{1 + x^2}} \label{eq:cosarctan} \\
      \sin(\arccos x) &= & \sqrt{1 - x^2}           \label{eq:sinarccos} \\
      \sin(\arcsin x) &= &  x                       \label{eq:sinarcsin} \\
      \sin(\arctan x) &= & \frac{x}{\sqrt{1 + x^2}} \label{eq:sinarctan} \\
      \tan(\arccos x) &= & \frac{\sqrt{1 - x^2}}{x} \label{eq:tanarccos} \\
      \tan(\arcsin x) &= & \frac{x}{\sqrt{1 - x^2}} \label{eq:tanarcsin} \\
      \tan(\arctan x) &= & x                        \label{eq:tanarctan}
  \end{eqnarray} 

  %\begin{equation}
  %  \begin{array}{r@=l r@=l r@=l}
  %    \cos(\arccos x) & x 
  %    & \cos(\arcsin x) & \sqrt{1 - x^2}
  %    & \cos(\arctan x) & \frac{1}{\sqrt{1 + x^2}} \\
  %    \sin(\arccos x) & \sqrt{1 - x^2}
  %    & \sin(\arcsin x) &  x
  %    & \sin(\arctan x) & \frac{x}{\sqrt{1 + x^2}} \\
  %    \tan(\arccos x) & \frac{\sqrt{1 - x^2}}{x}
  %    & \tan(\arcsin x) & \frac{x}{\sqrt{1 - x^2}}
  %    & \tan(\arctan x) & x
  %  \end{array}
  %\end{equation}



\section{Generalities}

  A projection is a transformation 
  $F: \mathbb{R}^2 \to \mathbb{R}^2$,
  $(X, Y) = F(\alpha, \delta)$
  \begin{equation*}
    F:
    \begin{pmatrix}
      \alpha \\
      \delta
    \end{pmatrix}
    \to
    \begin{pmatrix}
      X = f_1(\alpha, \delta) \\
      Y = f_2(\alpha, \delta) \\
    \end{pmatrix}
  \end{equation*}
  The associated differential transform is
  \begin{equation}
    \mathrm{d}X\mathrm{d}Y = |\det J_F(\alpha,\delta)| \mathrm{d}\alpha\mathrm{d}\delta
  \end{equation}
  with $J_F$ the Jacobian of the transformation $F$:
  \begin{equation}
    J_F = 
    \begin{pmatrix}
      \frac{\partial f_1}{\partial \alpha} & \frac{\partial f_1}{\partial \delta} \\
      \frac{\partial f_2}{\partial \alpha} & \frac{\partial f_2}{\partial \delta}
    \end{pmatrix}
  \end{equation}
  A projection is equiareal if elementary surfaces are proportional, so if
  \begin{equation}
    \mathrm{d}X\mathrm{d}Y \propto \cos\delta\mathrm{d}\alpha\mathrm{d}\delta
    \label{eq:equiareal}
  \end{equation}
  So if $|\det J_F(\alpha,\delta)| \propto \cos\delta$.

  \subsection{Changing the center of the projection}

    By default we consider in this document that the centre of the projection
    is the vernal point, so the point of Eulidean coordinates $(1, 0, 0)$.

    To set the projection cetner to $(\alpha_c, \delta_c)$, simpliy mutliply the $(x, y, z)$
    coordinate of a celestial point by the rotation matrix:
    \begin{equation*}
      \begin{pmatrix}
        \cos\alpha_c\cos\delta_c &  \sin\alpha_c\cos\delta_c & \sin\delta_c\\
       -\sin\alpha_c             &  \cos\alpha_c             & 0 \\
       -\cos\alpha_c\sin\delta_c & -\sin\alpha_c\sin\delta_c & \cos\delta_c
      \end{pmatrix}
    \end{equation*}
    before the projection.

    And by the transpose of the matrix after the deprojection.

\section{Zenithal (or azimuthal) projections}

  Input:
  \begin{eqnarray}
    x & = & \cos(\rho) \label{eq:zen.x} \\
    y & = & \sin(\rho) \cos(\theta) \label{eq:zen.y} \\
    z & = & \sin(\rho) \sin(\theta) \label{eq:zen.z}
  \end{eqnarray}
  in which $\rho$ is the angular distance between the reference point $(1, 0, 0)$
  and the point to be projected $(x, y, z)$ and $\theta$ is the angle between
  $\vec{y} = \vec{(0, 1, 0)}$ and $\vec{(0, y, z)}$, i.e. $\theta=\frac{\pi}{2} - PA$, $PA$
  being the position angle (east of north) between the reference point and the point
  to be projected.

  Output:
  \begin{eqnarray}
    X & = & R \cos(\theta) \label{eq:zenithal.X} \\
    Y & = & R \sin(\theta) \label{eq:zenithal.Y}
  \end{eqnarray}
  
  We have the relations
  \begin{eqnarray}
    x^2 + y^2 + z^2  & = & 1 \\
    \sqrt{y^2 + z^2} & = & \sin(\rho) = r \\
              \theta & = & \arctan{\frac{y}{x}} \\
	      \theta & = & \arctan{\frac{Y}{X}} \\
                   R & = & \sqrt{X^2 + Y^2}
  \end{eqnarray}

  In the FITS reference paper \citep{Calabretta2002}, different conventions are used.
  They use $\phi_{cg}$ and $\theta_{cg}$ ($cg$ stands for ``Calabretta and Greisen'')
  wich are linked to our conventions by the formula:
  \begin{eqnarray}
    \theta & = & \frac{\pi}{2} - \frac{\pi}{180}\phi_{cg}   \label{eq:zen:theta} \\
    \rho   & = & \frac{\pi}{2} - \frac{\pi}{180}\theta_{cg} \label{eq:zen:rho}
  \end{eqnarray}
  Depending on if we look from the inner of the unit sphere or the outter of the unit sphere,
  we may also have
  \begin{equation}
    \theta = \frac{\pi}{2} - \frac{\pi}{180}\phi_{cg}   \label{eq:zen:theta_v2}
  \end{equation}


  \subsection{AZP: Zenithal perspective \label{sec:azp}}

  The point of projection $P$ is on the $x$-axis and its coordinates (in the local frame) are $(-\mu, 0, 0)$.
  The projection plane is not perpendicular to the $x$-axis but is tilted of an angle $\gamma$
  (angle between the perpendicular to the plane at the point (1, 0, 0) and the $x$-axis) in the $xz$ plane.
  \begin{itemize}
    \item $\mu$ - is the opposite of the abscissa of the projection point $P$ and must be $\ne -1$
          (in which case $P$ is on the projection plane)
    \item $\gamma$ - angle between the projection plane and the $x$-axis in the $xz$ plane.
  \end{itemize}
  In FITS, $\mu$ is provided by keyword PV$i\_1a$ (no units, default value $=0$) and
  $\gamma$ is provided by keyword PV$i\_2a$ (in degrees, default value $=0$).
  The projection is defined by the following equations:
  \begin{eqnarray}
    X & = & D \cos\theta \\
    Y & = & D \frac{\sin\theta}{\cos\gamma} \\
    D & = & \frac{(\mu + 1) \sin\rho}{(\mu + \cos\rho) - \sin\rho\sin\theta\tan\gamma}
  \end{eqnarray}
  Remark: $\mu + 1$ is the coordinate of $P$ from the point (1, 0, 0) and allong the opposite of the $x$-axis.

\subsubsection{Projection}

      We rewrite
      \begin{equation} 
        D = \frac{(\mu + 1) \sin\rho}{(\mu + x) - z\tan\gamma}
      \end{equation}
      So
      \begin{eqnarray}
        X & = & D \frac{y}{\sin\rho}           = \frac{y(\mu + 1)}{(\mu + x) - z\tan\gamma} \\
        Y & = & D \frac{z}{\sin\rho\cos\gamma} = \frac{z(\mu + 1)}{\cos\gamma(\mu + x) - z\sin\gamma}
      \end{eqnarray}
      The limit of the projection if $|\mu|>1$ is $x = \frac{-1}{\mu}$, so if $x < \frac{-1}{\mu}$ a
      point is not projected.
      From this value, we can compute the maximum value of $D$, which depends on $\theta$:
      \begin{eqnarray}
        D_{max}(\theta) & = & \frac{(\mu + 1)\sqrt{1 - \frac{1}{\mu^2}}}
	                           {(\mu - \frac{1}{\mu}) - \sqrt{1 - \frac{1}{\mu^2}}\sin\theta\tan\gamma} \\
                        & = & \frac{\mu + 1}{\sqrt{\mu^2 - 1} - \sin\theta\tan\gamma}
      \end{eqnarray}
      The limit of the projection if $|\mu|<1$ is more complex.
      The limit is the plane parallel to the projection plane and containing the point of position
      $(-\mu, 0, 0)$.
      We can then write the limit: $(x, y, z).(-\cos\gamma, 0, \sin\gamma) <= \mu\cos\gamma$, which gives
      \begin{equation}
        (x+\mu)\cos\gamma \le z\sin\gamma
      \end{equation}
      we do not use $\tan\gamma$ not to have to change the inequality according to the sign of $\cos\gamma$
      and to be still able to use the formula when $\gamma\approx\pm \pi/2$.

\subsubsection{Deprojection}

      We have to revert $D$ to obtain $\rho$.
      \begin{eqnarray}
        D & = & \sqrt{X^2 + Y^2\cos^2\gamma} \\
	\sin\theta & = & \frac{Y\cos\gamma}{D} \\
	D(\mu + \cos\rho) - \sin\rho Y\sin\gamma & = & (\mu + 1)\sin\rho \label{eq:szp:deproj:d}
      \end{eqnarray}
      First of all, we compute the distance max to be sure the given point are in the limits of the projection:
      \begin{equation}
        D_{max}(Y) = \frac{\mu + 1}{\sqrt{\mu^2 - 1} - \frac{Y}{D}\sin\gamma}
      \end{equation}
      So if $D \le D_{max}(Y)$ or ,equivalently, if $D\sqrt{\mu^2+1}\le (\mu + 1) + Y\sin\gamma$, we continue.
      Dividing by $D$ on both sides of Eq. (\ref{eq:szp:deproj:d}), we rewrite it:
      \begin{equation}
        \frac{(\mu + 1) + Y\sin\gamma}{D}\sin\rho - \cos\rho = \mu
      \end{equation}
      If $(\mu + 1) = -Y\sin\gamma$, the equation reduces to $\cos\rho = -\mu$ so $\rho=\arccos(-\mu)$ (which is possible only of $\mu \in [-1, 1]$.
      To better handle small distance (i.e. $D\approx 0$) we rewrite the previous equation:
      \begin{equation}
        \sin\rho - \frac{D}{(\mu + 1) + Y\sin\gamma}\cos\rho = \frac{\mu D}{(\mu + 1) + Y\sin\gamma}
      \end{equation}
      If $D=0$, it reduces to $\sin\rho=0$ so to $\rho=0$ (can't be $\pi$ since the opposite point is hidden in htis projection).
      Previous equation is an equation of the form:
      \begin{equation}
        A\sin a - B\cos a = C
      \end{equation}
      That we solve rewriting it
      \begin{equation}
        \frac{A}{\sqrt{A^2 + B^2}}\sin a - \frac{B}{\sqrt{A^2 + B^2}}\cos a = \frac{C}{\sqrt{A^2 + B^2}}
      \end{equation}
      and using the equality ($A/\sqrt{A^2 + B^2}$ and $B/\sqrt{A^2+B^2}$ are $\in [0, 1]$ so they can be the value if sines or cosines)
      \begin{equation}
        \sin(a-b) = \sin a\cos b - \cos a\sin b
      \end{equation}
      Leading to
      \begin{equation}
        a = \arcsin \frac{C}{\sqrt{A^2 + B^2}} + \arctan(B/A) 
      \end{equation}
      Because of the $\arcsin$ function which results are $\in [-\pi/2, \pi/2]$, in the case of $|\mu|<1$ we may choose the solution
      \begin{equation}
        a = \pi - \arcsin \frac{C}{\sqrt{A^2 + B^2}} + \arctan(B/A) 
      \end{equation}
      So in our case in which $C=\mu B$, we have 
      \begin{equation}
        \rho = \arcsin \frac{\mu B}{\sqrt{B^2 + 1}} + \arctan B
      \end{equation}
      with
      \begin{equation}
        B = \frac{D}{(\mu + 1) + Y\sin\gamma}
      \end{equation}
      In case of $|\mu|<1$, we have the solution:
      \begin{equation}
        \rho =
	\begin{cases}
	  \arcsin \frac{\mu B}{\sqrt{B^2 + 1}} + \arctan B       &\text{if $\sin(\rho) > 0$} \\
	  \pi - \arcsin \frac{\mu B}{\sqrt{B^2 + 1}} + \arctan B &\text{if $\sin(\rho) < 0$}
        \end{cases}
      \end{equation}
      But we are interested in $sin\rho$ and $\cos\rho$,
      and trigonometric functions are costly (in term of CPU)  
      so using:
      \begin{equation}
        \sin(a+b) = \sin a\cos b + \cos a\sin b
      \end{equation}
      we can write
      \begin{eqnarray}
        \sin \rho & = & \sin\left(\arcsin \frac{\mu B}{\sqrt{B^2 + 1}} + \arctan B\right) \\
	          & = & \frac{\mu B}{\sqrt{B^2 + 1}}        \frac{1}{\sqrt{1 + B^2}}
		      + \sqrt{1 - \frac{\mu^2B^2}{B^2 + 1}} \frac{B}{\sqrt{1 + B^2}} \\
		  & = & (w_2 + w_3 B) / w_1
      \end{eqnarray}
      and
      \begin{eqnarray}
        \sin((\pi-a)+b) & = & \sin (\pi - a)\cos b + \cos(\pi - a)\sin b \\
	                & = & (\sin\pi\cos a - \cos\pi\sin a)\cos b + (\cos\pi\cos a + \sin\pi\sin a))\sin b \\
			& = & \sin a\cos b - \cos a\sin b \\
			& = & (w_2 - w_3B) / w_1
      \end{eqnarray}
      and using
      \begin{equation}
        \cos(a+b) = \cos a\cos b - \sin a\sin b
      \end{equation}
      we can write
      \begin{eqnarray}
        \cos \rho & = & \cos\left(\arcsin \frac{\mu B}{\sqrt{B^2 + 1}} + \arctan\sqrt{B^2 + 1}\right) \\
		  & = & \sqrt{1 - \frac{\mu^2B^2}{B^2 + 1}}  \frac{1}{\sqrt{1 + B^2)}}
		      - \frac{\mu B}{\sqrt{B^2 + 1}}         \frac{B}{\sqrt{1 + B^2}} \\
		  & = & (w_3 - w_2 B) / w_1
      \end{eqnarray}
      and
      \begin{eqnarray}
        \cos((\pi-a)+b) & = & \cos(\pi - a)\cos b - \sin(\pi - a)\sin b \\
			& = & -(\cos a\cos b + \sin a\cos b) \\
			& = & -(w_3 + w_2B) /w_1
      \end{eqnarray}
      with
      \begin{eqnarray}
        w_1 & = & \sqrt{B^2+1}  \\
        w_2 & = & \frac{\mu B}{w_1} \\
        w_3 & = & \sqrt{1 - w_2^2}
      \end{eqnarray}
      So at the end we have all ingredients to compute
      \begin{eqnarray}
        x & = & \cos\rho \\
	y & = & \sin\rho \frac{X}{D} \\
	z & = & \sin\rho \frac{Y\cos\gamma}{D}
      \end{eqnarray}


  \subsection{SZP: Slant zenithal perspective \label{sec:szp}}

  The projection plane is tangential to the sphere in $C=(1, 0, 0)$ and the 
  projection point $P$ is located at position $(x_p, y_p, z_p)$ in the local frame
  (i.e. the frame centered at the center $O$ of the unit sphere).
  In FITS \citep[see][\S 5.1.2]{Calabretta2002}, the position of $P$ is given by three parameters:
  \begin{itemize}
    \item $\mu$ - distance $OP$ (PV$i\_1a$, no units, default = 0), negative if $P$ is on the side of planewards hemisphere, positive otherwise
    \item $\phi_c$   - angle (PV$i\_2a$, in degrees, default = 0) between the opposite of the $Y$-axis and the projection of $P$ onto the $XY$-plane)
    \item $\theta_c$ - angle (PV$i\_3a$, in degrees, default = 90) between the $y$-axis rotated of the angle $\frac{\pi}{2} - \frac{\pi}{180}\phi_c$ around the $x$-axis
                       and the point of intersection between $PO$ and the unit sphere on the planewards side (planewards hemisphere)
  \end{itemize}
  In our framework, we prefer using an always positive distance $r_p$:
  \begin{itemize}
    \item $r_p$ - distance (no units) of the point $P$ to the center of the sphere, always positive
    \item $\theta_p$ - angle (in radian, default =  $\pi/2$) between the $X$-axis and the segment joining $C$ and the projection of $P$ on the $XY$-plane
    \item $\rho_p$   - angular distance (in radians, default = 0) between the point $C$ and the intersecition of $PO$ with the unit sphere
  \end{itemize}
  So that the coordinates of $P$ are:
  \begin{eqnarray}
     x_p & = & r_p \cos\rho_p \\
     y_p & = & r_p \sin\rho_p \cos\theta_p\\
     z_p & = & r_p \sin\rho_p \sin\theta_p
  \end{eqnarray}
  Using longitude $l_r$ and latitude $b_r$:
  \begin{eqnarray}
     x_p & = & r_p \cos b_p \cos l_p \\
     y_p & = & r_p \cos b_p \sin l_p\\
     z_p & = & r_p \sin b_p
  \end{eqnarray}
  so that the relation betwee both systems are:
  \begin{eqnarray}
    \theta_p & = & \arctan{ \frac{\sin b_p}{\cos b_p \sin l_p} } \\
    \rho_p & = & \arccos{ \cos b_p \cos l_p }
  \end{eqnarray}

  The relations between the FITS conventions and this conventions are (see Eq. \ref{eq:zen:theta} and \ref{eq:zen:rho}):
  \begin{eqnarray}
    r_p & = & |\mu| \\
    \theta_p & = & \frac{\pi}{2} - \frac{\pi}{180}\phi_c     \\
    \rho_p & = & 
        \begin{cases}
           \frac{\pi}{2} - \frac{\pi}{180}\theta_c,& \text{if } \mu < 0 \\
           \frac{\pi}{2} + \frac{\pi}{180}\theta_c,& \text{otherwise}
        \end{cases}
  \end{eqnarray}
  Again, according to if we look from the inside or the outside of the sphere
  we may consider
  \begin{equation}
    \theta_p = \frac{3\pi}{2} - \frac{\pi}{180}\phi_c
  \end{equation}
  %\begin{eqnarray}
  %  X & = & D \cos\theta \\
  %  Y & = & D \frac{\sin\theta}{\cos\gamma} \\
  %  D & = & \frac{(\mu + 1) \sin\rho}{(\mu + \cos\rho) - \sin\rho\sin\theta\tan\gamma}
  %\end{eqnarray}

\subsubsection{Projection}

  We consider a point $A$ of coordinates $(x, y, z)$ on the unit sphere
  and $A'$ of coordinates $(1, X, Y)$ its projection on the projection plane such that $PAA'$ are on a same straitgh line.
  We note $P'$ of coordinate $(1, y_p, z_p)$ the projection of $P$ onto the projection plane such that $PP'$ is perpendicular to the projection plane.
  We also note $A''$ the projection of $A$ on $PP'$ such that $AA''\perp PP'$.\\
  If $r_p \in [0, 1]$, i.e. the projection point is inside the unit sphere, we can project all point suchs that $x > x_p$.
  If $r_p > 1$ and $\rho_r > \pi / 2$ ($\Leftrightarrow x_p >0$), then hidden points are nearest points from $P$ which are in the cone of center $P$ and delimited by
  the tangents of the unit sphere passing through $P$. We note $T$ a point  of the unit sphere such taht $TP$ is tangantial to the sphere
  an $T'$ the projection of $T$ onto $PO$ (such that the triangle $PT'T$ is right in $T'$.
  Angle $\sin\widehat{OPT} = \frac{1}{r_p}$ and $\widehat{TOP}=\frac{\pi}{2} - \widehat{OPT}$.
  Hidden points are defined such that:
  \begin{equation*}
    \frac{\vec{OA}.\vec{OP}}{||\vec{OP}||} > \cos\widehat{TOP} = \sin\widehat{OPT}
  \end{equation*}
  I.e.
  \begin{equation}
    xx_p + yy_p + zz_p > 1 \label{eq:szp:proj:bounds1}
  \end{equation}
  If $r_p > 1$ and $\rho_r < \pi / 2$ ($\Leftrightarrow x_p <0$), the result is the opposite: only the nearest point
  from $P$ which are in the cone of center $P$ and delimited by
  the tangents of the unit sphere passing through $P$ are projected.
  Hidden points are defined such that:
  \begin{equation}
    xx_p + yy_p + zz_p < 1 \label{eq:szp:proj:bounds2}
  \end{equation}
  We can write this using the XOR operator ($\hat{\;}$), so points are hidden is the following expression is true:
  \begin{equation*}
    \left[(r_p \le 1)\:\&\&\:(x \le x_p)\right]\:||\:\left[(r_p > 1)\:\&\&\:\left( (x_p < 0) \hat{\;} (xx_p + yy_p + zz_p > 1)\right)\right] 
  \end{equation*}

  We now want to compute the projected positions of valid points considering the $xz$-plane in a first step.
  We use the triangle $A_{xz}'P_{xz}P_{xz}'$ and $A_{xz}P_{xz}A_{xz}''$ (we use notation $_{xz}$ to designate the projections on the $xz$ plane).
  \begin{eqnarray*}
    \tan\widehat{A_{xz}'P_{xz}P_{xz}'} & = & \frac{Y - z_p}{1 - x_p} \\
    \tan\widehat{A_{xz}P_{xz}A_{xz}''} & = & \frac{z - z_p}{x - x_p}
  \end{eqnarray*}
  As the two angle are equals, we easily express $Y$ (and similarly for $X$):
  \begin{eqnarray}
    X & = & (1 - x_p)\frac{y - y_p}{x - x_p} + y_p = \frac{y (1 - x_p) - y_p (1 - x)}{x - x_p} \label{eq:szp:X} \\
    Y & = & (1 - x_p)\frac{z - z_p}{x - x_p} + z_p = \frac{z (1 - x_p) - z_p (1 - x)}{x - x_p} \label{eq:szp:Y}
  \end{eqnarray}

\subsubsection{Deprojection}

  We introduces the following simplifying notations from the previous section:
  \begin{eqnarray*}
    T_X & = & \tan\widehat{A_{xy}'P_{xy}P_{xy}'} = \frac{X - y_p}{1 - x_p} \\ 
    T_Y & = & \tan\widehat{A_{xz}'P_{xz}P_{xz}'} = \frac{Y - z_p}{1 - x_p}
  \end{eqnarray*}
  We need first to compute $x$ from previous equations Eq. \ref{eq:szp:X} and \ref{eq:szp:Y} removing $y$ and $z$ 
  terms remembering that $y^2 + z^2 = 1 - x^2$.
  \begin{eqnarray*}
    y & = & T_X (x - x_p) + y_p = x T_X - (T_X x_p - y_p) \\
    z & = & T_Y (x - x_p) + z_p = x T_Y - (T_Y x_p - z_p) 
  \end{eqnarray*}
  We introduce notations:
  \begin{eqnarray*}
    T_X' & = & T_X x_p - y_p \\
    T_Y' & = & T_Y x_p - z_p
  \end{eqnarray*}
  We now take the sum of the squares of both previous equations (in $y$ and $z$) to obtain an equation of second king in $x$ (remembering that $y^2 + z^2 = 1 - x^2$):
  \begin{eqnarray*}
    x^2(T_X^2 + T_Y^2 + 1) -2x(T_X T_X' + T_Y T_Y')  + (T_X'^2 + T_Y'^2 - 1) & = & 0 \\
  \end{eqnarray*}
  So
  \begin{equation}
    x = \frac{-b\pm \sqrt{b^2 - 4ac}}{2a}
  \end{equation}
  with
  \begin{eqnarray*}
    a & = & T_X^2 + T_Y^2 + 1 \\
    b & = & -2(T_X T_X' + T_Y T_Y')  \\
    c & = & T_X'^2 + T_Y'^2 - 1
  \end{eqnarray*}
  If we divide all three coefficients $a$, $b$, and $c$ by 2, we find the same result as in \cite{Calabretta2002}:
  \begin{equation*}
    x = \frac{-b'\pm \sqrt{b'^2 - ac}}{a}
  \end{equation*}
  with $b' = b/2$.
  We keep the highest $x$ (that must be $\in [1, -1]$, i.e. following solution (and we derive easily $y$ and $z$
  from the expressions of $T_X$ and $T_Y$):
  \begin{eqnarray}
    x & = & \frac{-b + \sqrt{b^2 - 4ac}}{2a} \\
    y & = & T_X(x - x_p) + y_p \\
    z & = & T_Y(x - x_p) + z_p
  \end{eqnarray}

  About validity if $r>1$.
  Coordiantes of a point in the projection plane are $(1, X, Y)$.
  The projection of the cone of vertex $P$ ($\vec{p}$), axe $PO$ and angle $\psi=\arcsin(1/r_p)$ (so that it is tangential to the unit sphere)
  on the projection plane is a conic section (so a circle, an ellipse, a parabola or an hyperbola).
  All point $\vec{v}$ inside the cone are such that (we perfrom a translation to place the origin of the frame at the vertex of cone):
  \begin{equation*}
    |\frac{(\vec{v}-\vec{p}).(\vec{-p})}{||\vec{v}-\vec{p}||\times ||\vec{-p}||}| \ge \cos\arcsin\frac{1}{r_p}
  \end{equation*}
  Thus, using Eq. (\ref{eq:cosarcsin}),
  \begin{eqnarray*}
    \frac{|\vec{p}^2-\vec{v}.\vec{p}|}{r_p\sqrt{\vec{v}^2+r_p^2-2\vec{v}.\vec{p}}} & \ge & \sqrt{1 - \frac{1}{r_p^2}} \\
    \frac{|r_p^2-\vec{v}.\vec{p}|}{\sqrt{\vec{v}^2+r_p^2-2\vec{v}.\vec{p}}} & \ge & \sqrt{r_p^2 - 1}
  \end{eqnarray*}
  or, written explicitly:
  \begin{equation}
    \frac{\left[ x_p(1 - x_p) + y_p(X - y_p) + z_p(Y - z_p) \right]^2}{(1 - x_p)^2 + (X - y_p)^2 + (Y - z_p)^2} \ge r^2 - 1
  \end{equation}
  Remark: in the projection part, $\vec{v}^2=1$, so we could have use instead of Eq. (\ref{eq:szp:proj:bounds1}) and (\ref{eq:szp:proj:bounds2})
  \begin{eqnarray*}
    \frac{|r_p^2-\vec{v}.\vec{p}|}{\sqrt{1+r_p^2-2\vec{v}.\vec{p}}} & \ge & \sqrt{r_p^2 - 1}
  \end{eqnarray*}
  i.e.
  \begin{equation}
    \frac{\left[ x_p(x - x_p) + y_p(y - y_p) + z_p(z - z_p) \right]^2}{(x - x_p)^2 + (y - y_p)^2 + (z - z_p)^2} \ge r^2 - 1
  \end{equation}
  but developping this inequation we find the simple forms  Eq. (\ref{eq:szp:proj:bounds1}) and (\ref{eq:szp:proj:bounds2})
  % (r^2 - v.p) / (1 -2v.p + r^2) >= r^2 - 1


  \subsection{TAN: Gnomonic}

  The Gnomonic projection is a special case of zenithal perspective
  projection (\S \ref{sec:azp}) with $\mu=0$ and $\gamma=0$.

  \subsubsection{Projection}

    We simply apply Eq. (\ref{eq:zenithal.X}) and  (\ref{eq:zenithal.Y})
    with
    \begin{equation}
      R = \tan\rho
    \end{equation}
    We project the hemisphere such that $\rho\le\pi/2$ on the full $XY$-plane
    with divergences for $\rho=\pi/2$, thus for $x=0$.

    Noting that
    \begin{equation}
      R = \frac{\sin\rho}{\cos\rho} = \frac{\sin\rho}{x},
    \end{equation}
    and using Eq. (\ref{eq:zenithal.X}) and (\ref{eq:zenithal.Y}) we obtain
    the projection formulea:
    \begin{eqnarray}
      X & = & \frac{y}{x}, \label{eq:tan.proj.X} \\
      Y & = & \frac{z}{x}. \label{eq:stg.proj.Y}
    \end{eqnarray}

  \subsubsection{Deprojection}

    We have
    \begin{equation}
      R^2 = X^2 + Y^2 = \frac{r^2}{x^2} = \frac{1 - x^2}{x^2},
    \end{equation}
    which leads to
    \begin{equation}
      x = \frac{1}{\sqrt{1 + X^2 + Y^2}}
    \end{equation}
    and using Eq. (\ref{eq:tan.proj.X}) and (\ref{eq:stg.proj.Y}) we obtain
    \begin{eqnarray}
      y & = & xX \\
      z & = & xY
    \end{eqnarray}


  \subsection{STG: Stereographic}

  The Stereographic projection is a special case of zenithal perspective
  projection (\S \ref{sec:azp}) with $\mu=1$ and $\gamma=0$.
  
  \subsubsection{Projection}

    We simply apply Eq. (\ref{eq:zenithal.X}) and  (\ref{eq:zenithal.Y})
    with
    \begin{equation}
      R = 2\tan\frac{\rho}{2}.
    \end{equation}
    We project the full sphere on the full $XY$-plane with a divergent point at
    $(\alpha, \delta) = (180, 0)$,  i.e. for $x=-1$.

    Using Eq. (\ref{eq:tanx2}) we note
    \begin{eqnarray}
      R & = & 2\frac{\sin\rho}{1 + \cos\rho}, \\
        & = & 2\frac{\sin\rho}{1 + x},
    \end{eqnarray}
    which leads to the projection formulae:
    \begin{eqnarray}
      X & = & \frac{2y}{1 + x}; \label{eq:stg.proj.X} \\
      Y & = & \frac{2z}{1 + x}. \label{eq:stg.proj.Y}
    \end{eqnarray}

  \subsubsection{Deprojection}
     
    From the two equiations
    \begin{eqnarray}
      R^2 & = & X^2 + Y^2 = \frac{4(y^2 + z^2)}{(1 + x)^2}, \\
      1 & = & x^2 + y^2 + z^2,
    \end{eqnarray}
    we deduce
    \begin{equation}
      R^2 = \frac{4(1 - x^2)}{(1 + x)^2} = \frac{4(1 - x)}{(1 + x)},
    \end{equation}
    That we solve to obtain $x$:
    \begin{equation}
      x = \frac{4 - R^2}{4 + R^2}
    \end{equation}
    From the projection formulae Eq. (\ref{eq:stg.proj.X}) and (\ref{eq:stg.proj.Y}) we deduce
    \begin{eqnarray}
      y & = & \frac{1}{2}X(1 + x) = \frac{4X}{4 + R^2}\\
      z & = & \frac{1}{2}Y(1 + x) = \frac{4Y}{4 + R^2}
    \end{eqnarray}
    We finally find the deprojection formulae:
    \begin{eqnarray}
      R' & = & \frac{X^2 + Y^2}{4}, \\
      w & = & \frac{1}{1 + R'}, \\
      x & = & w(1-R'), \label{eq:stg.deproj.x} \\
      y & = & wX, \label{eq:stg.deproj.y}\\
      z & = & wY. \label{eq:stg.deproj.z}
    \end{eqnarray}




  \subsection{SIN: Slant orthographic}

  It is the slant zenithal perspective (SZP) projection with the distance 
  between the centre of the sphere and the projection point $r_p$ equal to $+\infty$.

  \subsubsection{Projection}

    We use the limit of the projection equations of SZP, that is Eq. (\ref{eq:szp:X}) and (\ref{eq:szp:Y}),
    when $r_p\to+\infty$:
    \begin{eqnarray}
      X & = & \frac{-yx_p-y_p(1-x)}{-x_p} = y + \xi_p(1-x) \label{eq:sin.proj.X} \\
      Y & = & z + \eta_p (1-x) \label{eq:sin.proj.Y}
    \end{eqnarray}
    with
    \begin{eqnarray}
      \xi_p  & = & \frac{y_p}{x_p} = \tan\rho_p \cos\theta_p \\
      \eta_p & = & \frac{z_p}{x_p} = \tan\rho_p \sin\theta_p
    \end{eqnarray}
    The unitless parameters $\xi$ and $\eta$ are provided by the FITS keywords $PVi\_1a$ and $PVi\_2a$
    respectively. Both their default values equal zero. But in the FIST definition, 
    \begin{eqnarray}
      \xi  & = & \frac{\cos\theta_c}{\sin\theta_c}\sin\phi_c \\
      \eta & = & -\frac{\cos\theta_c}{\sin\theta_c}\cos\phi_c\
    \end{eqnarray}
    with the relation between $(\rho_p, \theta_p)$ and $(\theta_c, \phi_c)$ given in \S \ref{sec:szp},
    leading to (case $\mu>0$)
    \begin{eqnarray}
      \xi_p & = & -\xi \\
      \eta_p & = & \eta
    \end{eqnarray}
    Demonstration:
    \begin{eqnarray}
      \xi_p & = & \tan\rho_p \cos\theta_p \\
            & = & \frac{\sin(\frac{\pi}{2} + \theta_c)}{\cos(\frac{\pi}{2} + \theta_c)} \cos(\frac{\pi}{2} - \phi_c) \\
	    & = & \frac{\sin(\frac{\pi}{2} -(-\theta_c))}{\cos(\frac{\pi}{2} - (-\theta_c))} \cos(\frac{\pi}{2} - \phi_c) \\
	    & = & \frac{\cos(-\theta_c)}{\sin(-\theta_c)} \sin(\phi_c) \\
	    & = & \frac{\cos(\theta_c)}{-\sin(\theta_c)}\sin(\phi_c) \\
	    & = & -\xi \\ 
            %& = & \frac{\sin(\frac{\pi}{2} + \theta_c)}{\cos(\frac{\pi}{2} + \theta_c)} \cos(\frac{3\pi}{2} - \phi_c) \\
	    %& = & \frac{\sin(\frac{\pi}{2} -(-\theta_c))}{\cos(\frac{\pi}{2} - (-\theta_c))} \cos(\pi + \frac{\pi}{2} - \phi_c) \\
	    %& = & \frac{\cos(-\theta_c)}{\sin(-\theta_c)} (-\cos(\frac{\pi}{2} - \phi_c)) \\
	    %& = & \frac{\cos(\theta_c)}{-\sin(\theta_c)} (-\sin(\phi_c)) \\
	    %& = & \frac{\cos(\theta_c)}{\sin(\theta_c)}\sin(\phi_c) \\
	    %& = & \xi \\
      \eta_p & = & \tan\rho_p \sin\theta_p \\
             & = & \frac{\sin(\frac{\pi}{2} + \theta_c)}{\cos(\frac{\pi}{2} + \theta_c)} \sin(\frac{\pi}{2} - \phi_c) \\
	     & = & \frac{\cos(\theta_c)}{-\sin(\theta_c)} \cos(\phi_c) \\
	     & = & \eta
             %& = & \frac{\sin(\frac{\pi}{2} + \theta_c)}{\cos(\frac{\pi}{2} + \theta_c)} \sin(\frac{3\pi}{2} - \phi_c) \\
	     %& = & \frac{\sin(\frac{\pi}{2} -(-\theta_c))}{\cos(\frac{\pi}{2} - (-\theta_c))} \sin(\pi + \frac{\pi}{2} - \phi_c) \\
	     %& = & \frac{\cos(-\theta_c)}{\sin(-\theta_c)} (-\sin(\frac{\pi}{2} - \phi_c)) \\
	     %& = & \frac{\cos(\theta_c)}{-\sin(\theta_c)} (-\cos(\phi_c)) \\
             %& = & \frac{\cos(\theta_c)}{\sin(\theta_c)}\cos(\phi_c) \\
	     %& = & -\eta
    \end{eqnarray}

    From $(\xi_p, \eta_p)$ we can deduce:
    \begin{eqnarray}
      \tan\rho_p & = & -\sqrt{\xi_p^2 + \eta_p^2} \\
      \rho_p & = & \pi + \arctan(-\sqrt{\xi_p^2 + \eta_p^2}) = \pi - \arctan(\sqrt{\xi_p^2 + \eta_p^2}) \\
      \theta_p & = & \arctan2(\frac{\eta_p}{-\sqrt{\xi_p^2 + \eta_p^2}}, \frac{\xi_p}{-\sqrt{\xi_p^2 + \eta_p^2}}) \\
      \cos\theta_p & = & -\frac{\xi_p}{\sqrt{\xi_p^2 + \eta_p^2}} \\
      \sin\theta_p & = & -\frac{\eta_p}{\sqrt{\xi_p^2 + \eta_p^2}}
    \end{eqnarray}
    We know that $\rho_p \in [\pi/2, \pi]$, so $\tan\rho_p$ is always negative.
    So the right solution of $\tan^2\rho_p = \xi_p^2 + \eta_p^2$ is the value $-\sqrt{\xi_p^2 + \eta_p^2}$.
    The $\arctan$ function returns value $\in ]-\pi/2, \pi/2[$ and is periodic of period $\pi$, 
    thus $\rho_p =  \pi + \arctan(-\sqrt{\xi_p^2 + \eta_p^2})$.
    We can thus deduce the coordiante of $P$ on the unit sphere:
    \begin{eqnarray}
      x_p & = & \cos\rho_p \\ 
          & = & \cos(\pi - \arctan(\sqrt{\xi_p^2 + \eta_p^2})) \\ %-\sqrt{\frac{\xi_p^2 + \eta_p^2}{1 + \xi_p^2 + \eta_p^2}}\\
	  & = & -\cos \arctan(\sqrt{\xi_p^2 + \eta_p^2}) \\
	  & = & \frac{-1}{\sqrt{1 + \xi_p^2 + \eta_p^2}} \\
      y_p & = & \sin\rho_p \cos\theta_p \\ %= \frac{1}{\sqrt{1 + \xi_p^2 + \eta_p^2}}\frac{\xi_p}{\sqrt{\xi_p^2 + \eta_p^2}} \\
          & = & \sin(\pi - \arctan(\sqrt{\xi_p^2 + \eta_p^2})) \frac{-\xi_p}{\sqrt{\xi_p^2 + \eta_p^2}} \\
	  & = & \sin(\arctan(\sqrt{\xi_p^2 + \eta_p^2})) \frac{-\xi_p}{\sqrt{\xi_p^2 + \eta_p^2}} \\
	  & = & \frac{\sqrt{\xi_p^2 + \eta_p^2}}{\sqrt{1 + \xi_p^2 + \eta_p^2}} \frac{-\xi_p}{\sqrt{\xi_p^2 + \eta_p^2}} \\
	  & = & \frac{-\xi_p}{\sqrt{1 + \xi_p^2 + \eta_p^2}} \\
      z_p & = & \sin\rho_p \sin\theta_p \\ %= \frac{1}{\sqrt{1 + \xi_p^2 + \eta_p^2}}\frac{\eta_p}{\sqrt{\xi_p^2 + \eta_p^2}}
          & = & \frac{\sqrt{\xi_p^2 + \eta_p^2}}{\sqrt{1 + \xi_p^2 + \eta_p^2}} \frac{-\eta_p}{\sqrt{\xi_p^2 + \eta_p^2}} \\
	  & = & \frac{-\eta_p}{\sqrt{1 + \xi_p^2 + \eta_p^2}}
    \end{eqnarray}
    To find this result, we use Eq. (\ref{eq:cosarctan}) and (\ref{eq:sinarctan}).
    We verify that $x_p^2 + y_p^2 + z_p^2 = 1$.\\
    The projection ``rays'' is a cylinder of axis $(x_p, y_p, z_p)$ and radius equal to one.
    The projected poitn of the sphere are the points which are the nearest from the projection plane.
    We thus dedeuce that we can project a point only if it scalar product with $(x_p, y_p, z_p)$ is negative
    \begin{equation}
      x x_p + y y_p + z z_p \le 0,
    \end{equation}
    the point of the sphere such as $x x_p + y y_p + z z_p = 0$ beigin at the edge.
    They form an ellipse on the projection plane (intersection between a plane and a cylinder).

  \subsubsection{Deprojection}

    We rewrite Eq. (\ref{eq:sin.proj.X}) and (\ref{eq:sin.proj.Y}) to obtain an expression of $y$ and $z$:
    \begin{eqnarray}
      y & = & X -  \xi_p (1-x) \label{eq:sin.deproj.y}\\
      z & = & Y - \eta_p (1-x) \label{eq:sin.deproj.z}
    \end{eqnarray}
    We sum their square (remembering that $y^2+z^2 = 1 - x^2$ to obtain a quadratic equation in $x$:
    \begin{eqnarray}
      y^2+z^2 & = & X^2 + Y^2 + (\xi_p^2 + \eta_p^2)(1-x)^2 - 2(1-x)(\xi X + \eta Y) \\
      1 - x^2 & = & x^2(\xi_p^2 + \eta_p^2) \nonumber \\
              &   & + 2x[(\xi X + \eta Y) - (\xi_p^2 + \eta_p^2)] \nonumber \\
              &   & + (X^2 + Y^2) + (\xi_p^2 + \eta_p^2) - 2(\xi X + \eta Y) \\
	    0 & = & x^2(1 + \xi_p^2 + \eta_p^2) \nonumber \\
	      &   & + 2x[(\xi X + \eta Y) - (\xi_p^2 + \eta_p^2)] \nonumber \\
	      &   & + (X^2 + Y^2) + (\xi_p^2 + \eta_p^2) - 2(\xi X + \eta Y)  - 1
    \end{eqnarray}
    Thus
    \begin{equation}
      x = \frac{-b+\sqrt{b^2-4ac}}{2a} \label{eq:sin.deproj.x}
    \end{equation}
    since we keep the value having the largest $x$, so (nearest from the projection plane).
    In the previous equation the constants are:
    \begin{eqnarray}
      \tan^2\rho_p & = & \xi_p^2 + \eta_p^2 \\
      R^2 & = & X^2 + Y^2 \\
      R' & = & \xi X + \eta Y \\
      a & = & (1 + \tan^2\rho_p);  \\
      b & = & 2(R' -  \tan^2\rho_p); \\
      c & = & R^2 - 2R' + \tan^2\rho_p - 1.
    \end{eqnarray}
    We then deduce $y$ anf $z$ from Eq. (\ref{eq:sin.deproj.y}) and (\ref{eq:sin.deproj.z}).

    Testing the validity of a projection point to be de-projected:
    We call
    $M$ a point on the projection plane of coordinates $(1, X, Y)$,
    $O$ the center of the unit sphere,
    $P$ the point on the unit sphere of coordinates $(x_p, y_p, z_p)$ and
    $A$ the projection of $M$ on the plane perpendicular to $\vec{OP}$ passing through $O$.
    We have:
    \begin{equation}
      \vec{OM} = (\vec{OM}.\vec{OP})\vec{OP} + \vec{OA}
    \end{equation}
    Thus
    \begin{equation}
      \vec{OA} =
      \begin{pmatrix}
        1 \\
	X \\
	Y
      \end{pmatrix}
      - (x_p + y_p X + z_p Y)
      \begin{pmatrix}
        x_p \\
	y_p \\
	z_p
      \end{pmatrix}
    \end{equation}
    The point is in the projection area if $||\vec{OA}||\le1$, so if
    \begin{equation}
      (1 - x_p s)^2 + (X - y_p s)^2 + (Y - z_p s)^2 \le 1
    \end{equation}
    noting $s=x_p + y_p X + z_p Y$.



  \subsection{NCP: non-slant SIN with projection point located at the North Celestial Pole}

  Provided we set the projection center to $(0, \pi/2)$, the equation are then the same as
  the simple (non-slant) SIN projection, i.e.:
  \begin{equation}
    R = \sin\rho.
  \end{equation}
  
  \subsubsection{Projection}

    We simply have
    \begin{eqnarray}
      X & = & y \\
      Y & = & z 
    \end{eqnarray}

  \subsubsection{Deprojection}

    \begin{eqnarray}
      x & = & \sqrt{1 - (X^2 + Y^2)} \\
      y & = & X \\
      z & = & Y
    \end{eqnarray}

 
  \subsection{ARC: Zenithal equidistant}

  The Zenithal equidistant projection is the simple case in which:
  \begin{equation}
    R = \rho
  \end{equation}

  \subsubsection{Projection}

    From Eq. (\ref{eq:zenithal.X}), (\ref{eq:zenithal.Y}), (\ref{eq:zen.x}) and (\ref{eq:zen.y}),
    we simply have:
    \begin{eqnarray}
      X & = & \rho\cos\theta = \frac{\rho}{\sin\rho}y
              = \frac{\arcsin\sqrt{y^2 + z^2}}{\sqrt{y^2 + z^2}}y
              = y\arcsinc r \\
      Y & = & \rho\sin\theta = \frac{\rho}{\sin\rho}z
              = \frac{\arcsin\sqrt{y^2 + z^2}}{\sqrt{y^2 + z^2}}z
              = z\arcsinc r 
    \end{eqnarray}
    In practice for large angular distances (let us say $\rho>\pi/2$), we 
    replace $\arcsinc r$ by $\arccos(x) / r$ for a better numercial precision.

  \subsubsection{Deprojection}
  
    We easily inverse the previous equations:
    \begin{equation}
      X^2 + Y^2 = \rho^2.
    \end{equation}
    Using Eq. (\ref{eq:zen.x}), we deduce $x$
    \begin{equation}
      x = \cos\rho = \cos(\sqrt{X^2 + Y^2})
    \end{equation}
    and from Eq. (\ref{eq:zen.y}), (\ref{eq:zen.z}), (\ref{eq:zenithal.X}) and (\ref{eq:zenithal.Y})
    we deduce
    \begin{eqnarray}
      y & = & X \frac{\sin\rho}{\rho} = X \sinc(\sqrt{X^2 + Y^2}) \\  
      z & = & Y \frac{\sin\rho}{\rho} = Y \sinc(\sqrt{X^2 + Y^2})
    \end{eqnarray}


  \subsection{ZPN: Zenithal polynomial}

  The zenithal polynomial projection is a generalization of
  the zenithal equidistant projection in which the Eculidean
  distance is a polynomial of the angular distance
  \begin{equation}
    R = P(\rho) = p_0 + \rho(p_1 + \rho(p_2... + \rho(p_{n-1} + \rho p_n )...))
    \label{eq:zen.zpn.R}
  \end{equation}

  \subsubsection{Projection}

    \begin{eqnarray}
      r & = & \sin\rho = \sqrt{y^2 + z^2} \\
      X & = & P(\rho)\cos\theta = P(\rho)\frac{y}{\sin\rho} = P(\arcsin r)\frac{y}{r} \\
      Y & = & P(\rho)\sin\theta = P(\rho)\frac{z}{\sin\rho} = P(\arcsin r)\frac{z}{r}
    \end{eqnarray}
    In practice for large angular distances (let us say $\rho>\pi/2$), we 
    replace $\arcsin r$ by $\arccos(x)$ for a better numercial precision.

  \subsubsection{Deprojection}

    We use the Newton-Raphson's method to invert numerically Eq. (\ref{eq:zen.zpn.R})
    and find $\rho$ from $R=\sqrt{X^2 + Y^2}$:
    \begin{equation}
      \rho = P^{-1}(R=\sqrt{X^2 + Y^2}).
    \end{equation}
    To do so we first have to compute the derivative of $P'(\rho)$ of $P(\rho)$
    \begin{equation}
      R' = P'(\rho) = p_1 + \rho(2p_2 + \rho(3p_3... + \rho((n-1)p_{n-1} + \rho n p_n )...)),
    \end{equation}
    then we solve iteratively $P(\rho) - R = 0$ using the recursion formula
    \begin{equation}
      \rho_{i+1} = \rho_i - \frac{P(\rho_i) - \sqrt{X^2 + Y^2}}{P'(\rho_ i)}
    \end{equation}
    We deduce $x$:
    \begin{equation}
      x = \cos\rho
    \end{equation}
    and inverting the projection formulae, we find
    \begin{eqnarray}
      y & = & \sqrt{\frac{1 - x^2}{X^2 + Y^2}}X \\
      z & = & \sqrt{\frac{1 - x^2}{X^2 + Y^2}}Y
    \end{eqnarray}
      

  \subsection{ZEA: Zenithal equal-area}

  The zenithal equal-area projection is the quite simple case in which
  \begin{equation}
    R = 2\sin\frac{\rho}{2}.
  \end{equation}

 \subsubsection{Projection}

  We develop the expression of the Euclidean distance $R$
  using Eq. (\ref{eq:cos2x}) and then Eq. (\ref{eq:zen.x}):
  \begin{eqnarray}
    R^2 & = & 4\sin^2\frac{\rho}{2} \\
        & = & 4\frac{1 - \cos\rho}{2} \\
        & = & 2(1 - x)
  \end{eqnarray}
  So the projection formulae are
  \begin{eqnarray}
    X & = & \sqrt{2(1 - x)} \frac{y}{\sin\rho} \\
      & = & \sqrt{\frac{2(1 - x)}{1 - x^2}} y \\
      & = & \sqrt{\frac{2}{1+x}}y \label{eq:zea.proj.X}\\
    Y & = & \sqrt{2(1 - x)} \frac{z}{\sin\rho} \\
      & = & \sqrt{\frac{2}{1+x}}z \label{eq:zea.proj.Y}
  \end{eqnarray}

  \subsubsection{Deprojection}

    We find the expression of $x$ from $R^2=X^2+Y^2$:
    \begin{eqnarray}
      R^2 & = & \frac{2}{1+x}r \\
          & = & \frac{2}{1+x}(1-x^2) \\
          & = & \frac{2}{1+x}(1+x)(1-x) \\
          & = & 2(1-x)
    \end{eqnarray}
    So
    \begin{equation}
      x = 1 - \frac{R^2}{2} = 1 - \frac{X^2+Y^2}{2}
    \end{equation}
    We now reinject this expression of $x$ in the projection formulae Eq. (\ref{eq:zea.proj.X}) and (\ref{eq:zea.proj.Y})
    To obtain 
    \begin{eqnarray}
      y & = & \sqrt{1-\frac{R^2}{4}}X = \sqrt{1-\frac{X^2+Y^2}{4}}X \\
      z & = & \sqrt{1-\frac{R^2}{4}}Y = \sqrt{1-\frac{X^2+Y^2}{4}}Y
    \end{eqnarray}
 


  \subsection{AIR: Airy projection}

  In the Airy projection (c.f. book ``Flattening the Earth: Two Thousand Years of Map Projections'' by John P. Snyder):
  \begin{eqnarray}
    R & = & 2\left(\frac{\ln\frac{1}{\cos\frac{\rho}{2}}}{\tan\frac{\rho}{2}}
                + \tan\frac{\rho}{2} \frac{\ln\frac{1}{\cos\frac{\rho_b}{2}}}{\tan^2\frac{\rho_b}{2}}
        \right) \\
      & = & -2\left(\frac{\ln \cos\frac{\rho}{2}}{\tan\frac{\rho}{2}}
                + \tan\frac{\rho}{2} \frac{\ln \cos\frac{\rho_b}{2}}{\tan^2\frac{\rho_b}{2}} \right)
  \end{eqnarray}

  \subsubsection{Projection}

    We use the fact that:
    \begin{eqnarray}
      \cos\frac{\rho}{2} & = & \sqrt{\frac{1 + \cos\rho}{2}} \\
      \sin\frac{\rho}{2} & = & \sqrt{\frac{1 - \cos\rho}{2}} \\
      \tan\frac{\rho}{2} & = & \sqrt{\frac{1 - \cos\rho}{1 + \cos\rho}}
    \end{eqnarray}
    and that $x = \cos\rho$ to write
    %\begin{equation}
    %  R = \sqrt{\frac{1 + x}{1 - x}}\left(
    %        \ln(\frac{x+1}{2})
    %      + \frac{1 - x}{1 - x_b}\frac{1 + x_b}{1 + x}\ln(\frac{x_b+1}{2})
    %  \right) 
    %\end{equation}
    \begin{equation}
      R = -\sqrt{\frac{1 + x}{1 - x}} \ln(\frac{x+1}{2})
        - \sqrt{\frac{1 - x}{1 + x}} \frac{1 + x_b}{1 - x_b}\ln(\frac{x_b+1}{2})
    \end{equation}
    Thus the projection formulae are:
    \begin{eqnarray}
      X & = & R \frac{y}{\sqrt{y^2 + z^2}} = R \frac{y}{\sqrt{(1-x)(1+x)}} \\
        & = & -y\left(\frac{\ln(\frac{x+1}{2})}{1-x}
                 + \frac{1}{1+x}\frac{1 + x_b}{1 - x_b}\ln(\frac{x_b+1}{2}) \right) \\
      Y & = & -z\left(\frac{\ln(\frac{x+1}{2})}{1-x}
                 + \frac{1}{1+x}\frac{1 + x_b}{1 - x_b}\ln(\frac{x_b+1}{2}) \right)
    \end{eqnarray}

  \subsubsection{Deprojection}

    \begin{eqnarray}
      \sqrt{X^2 + Y^2}
        & = & \sqrt{(1-x)(1+x)}\left(\frac{\ln(\frac{x+1}{2})}{1-x}
              + \frac{1}{1+x}\frac{1 + x_b}{1 - x_b}\ln(\frac{x_b+1}{2}) \right) \\
        & = & R \\ 
        & = & \sqrt{\frac{1 + x}{1 - x}} \ln(\frac{x+1}{2})
        + \sqrt{\frac{1 - x}{1 + x}} \frac{1 + x_b}{1 - x_b}\ln(\frac{x_b+1}{2}) 
    \end{eqnarray}
    We use the Newton-Raphson's algorithm to solve the equation
    \begin{equation}
      \sqrt{\frac{1 + x}{1 - x}} \ln(\frac{x+1}{2})
        + \sqrt{\frac{1 - x}{1 + x}} \frac{1 + x_b}{1 - x_b}\ln(\frac{x_b+1}{2}) - \sqrt{X^2+Y^2} = 0
    \end{equation}
    The derivatived of the parts of the function $R(x)$ are:
    \begin{equation}
      \frac{\partial}{\partial x} \sqrt{\frac{1+x}{1-x}} = \frac{1}{(1-x)^2}\sqrt{\frac{1-x}{1+x}}
    \end{equation}
    \begin{equation}
      \frac{\partial}{\partial x} \sqrt{\frac{1-x}{1+x}} = -\frac{1}{(1+x)^2}\sqrt{\frac{1+x}{1-x}}
    \end{equation}
    \begin{equation}
      \frac{\partial}{\partial x} \ln(\frac{x+1}{2}) = \frac{1}{1+x}
    \end{equation} 
    We deduce the value of the derivative of $R(x)$:
    \begin{eqnarray}
      \frac{\partial}{\partial x} R(x) & = & \frac{1}{1+x}\sqrt{\frac{1+x}{1-x}} \\
        & & + \frac{1}{(1-x)^2}\sqrt{\frac{1-x}{1+x}}\ln(\frac{x+1}{2}) \\
        & & -\frac{1}{(1+x)^2}\sqrt{\frac{1+x}{1-x}} \frac{1 + x_b}{1 - x_b}\ln(\frac{x_b+1}{2}) \\
       R'(x) & = & \frac{1}{\sqrt{1-x^2}}\left(1 + \frac{\ln(\frac{x+1}{2})}{1-x}
                - \frac{1}{1+x}\frac{1 + x_b}{1 - x_b}\ln(\frac{x_b+1}{2})
              \right)
    \end{eqnarray}
    So we find $x$ iteratively:
    \begin{equation}
      x_{i+1} = x_i - \frac{R(x) - \sqrt{X^2 + Y^2}}{R'(x)}
    \end{equation}
    We deduce the two other coordinates:
    \begin{eqnarray}
      y & = & \frac{X}{\sqrt{X^2 + Y^2}}\sqrt{1 - x^2} \\
      z & = & \frac{Y}{\sqrt{X^2 + Y^2}}\sqrt{1 - x^2}
    \end{eqnarray}

    To start the iteration process, we have to start with a first approximation.
    We can make a very raw approximation replacing the full expression of $R$ by:
    \begin{equation}
      R = \ln(\frac{1}{2})(1 - x^2) + (1 - x) \frac{1 + x_b}{1 - x_b}\ln(\frac{x_b+1}{2})
    \end{equation}
    This is an equation of the form $a x^2 + b x + c = 0$, ... we keep value of $x \in ]-1, 1]$.


\section{Cylindrical}

  Input:
  \begin{eqnarray}
    x & = & \cos(\rho) \label{eq:zen.x} \\
    y & = & \sin(\rho) \cos(\theta) \label{eq:zen.y} \\
    z & = & \sin(\rho) \sin(\theta) \label{eq:zen.z}
  \end{eqnarray}
  in which $\rho$ is the angular distance between the reference point $(1, 0, 0)$
  and the point to be projected $(x, y, z)$ and $\theta$ is the angle between
  $\vec{y} = \vec{(0, 1, 0)}$ and $\vec{(0, y, z)}$, i.e. $\theta=\frac{\pi}{2} - PA$, $PA$
  being the position angle (east of north) between the reference point and the point
  to be projected.

  Output:
  \begin{eqnarray}
    X & = & F_X(\alpha) \label{eq:cylindrical.X} \\
    Y & = & F_Y(\delta) \label{eq:cylindrical.Y}
  \end{eqnarray}


  \subsection{CAR: Platte carr\'ee}

  This is certainly the simplest cylindrical projection.

  \subsubsection{Projection}

    The projection formulae are
    \begin{eqnarray}
      X & = & \alpha, \\
      Y & = & \delta.
    \end{eqnarray}
    So in our framework, 
    \begin{eqnarray}
      X & = & \arctan 2(y, x), \\
      Y & = & \arcsin z = \arctan 2(z, \sqrt(x^2 + y^2)).
    \end{eqnarray}
    For precision reasons, it is better not to use $\arcsin z$.


  \subsubsection{Deprojection}

    The de-projection formulae are
    \begin{eqnarray}
      \alpha & = & X, \\
      \delta & = & Y.
    \end{eqnarray}
    So in our framework,
    \begin{eqnarray}
      x & = & \cos Y \cos X, \\
      y & = & \cos Y \sin X, \\
      z & = & \sin Y.
    \end{eqnarray}


  \subsection{CEA: cylindrical equal area}

  The cylindrical equal area projection depdends on a parameter $\lambda$
  so it is conformal at latitudes $\delta_c = \pm \arccos\sqrt{\lambda}$.
  The projection formulae are
  \begin{eqnarray}
    X & = & \alpha, \\
    Y & = & \frac{\sin\delta}{\lambda}.
  \end{eqnarray}
  WARNING: $X$ and $Y$ must be multiplied by $180/\pi$ to be consistent
  with FITS WCS conventions.

  \subsubsection{Projection}
    
    Given the projection formula depending on $(\alpha, \delta)$, we deduce
    \begin{eqnarray}
      X & = & \arctan 2(y, x), \\
      Y & = & \frac{z}{\lambda}
    \end{eqnarray}

  \subsubsection{Deprojection}

    \begin{eqnarray}
      x & = & \sqrt{1 - z^2}\cos X \\
      y & = & \sqrt{1 - z^2}\sin Y \\
      z & = & \lambda Y
    \end{eqnarray} 



  \subsection{CYP: Cylindrical perspective}

  The cylindrical perpective projection depends on two parameters
  $\mu$ and $\lambda$.
  \begin{itemize}
    \item $\mu$ is the distance from the centre of the unit sphere to the projection point for a given $\alpha$;
    \item $\lambda$ is the radius of the cylinder the unit sphere is projected on.
  \end{itemize}  
  In other words, for a position $(\alpha, \delta)$ to be projected, the projection point
  as the coordinate $(\alpha, 0)$ at a distance $\mu$ of the centre of the unit sphere.

  The projection formulae are
  \begin{eqnarray}
    X & = & \lambda \alpha \\
    Y & = & \frac{\mu + \lambda}{\mu + \cos\delta} \sin\delta
  \end{eqnarray}

  \subsubsection{Projection}

    \begin{eqnarray}
      X & = & \lambda \arctan 2(y, x) \\
      Y & = & \frac{\mu + \lambda}{\mu + \sqrt{1 - z^2}} z
    \end{eqnarray}

  \subsubsection{Deprojection}

     






\section{Pseudocylindrical}

  \subsection{MOL: Mollweide's}

  The Mollweide's projection is defined such that the full sky is projected
  inside an ellipse of surface area equal to $4\pi$ (the surface area of the
  unit sphere), with the half width ($W$, semi-major axis) equal twice the
  half height size ($H$, semi-minor axis).
  And the $x$-axis correspond to the longitude $\alpha \in [0, 2\pi[$ with
  $\alpha = 0$ corresponding to $X = 0$, $\alpha=\pi$ to $X=W$,
  $\alpha=\pi+\varepsilon$ to $X=-W+\eta$ and $\alpha=2\pi-\varepsilon$ to $X=-\eta$.
  The base equations of are 
  \begin{eqnarray*}
    X & = & W \frac{\alpha}{\pi} \cos\gamma \\
    Y & = & H                    \sin\gamma
  \end{eqnarray*}
  with the semi-major and semi-minor axis $a$ and $b$ respectively, such that
  \begin{eqnarray*}
    a & = & W = 2H \\
    b & = & H
  \end{eqnarray*}
  and the surface area of the ellipse $\pi a b = 4\pi$ leading to $H=\sqrt{2}$
  and thus to:
  \begin{eqnarray}
    X & = & 2\sqrt{2} \frac{\alpha}{\pi} \cos\gamma \\
    Y & = &  \sqrt{2}                    \sin\gamma
  \end{eqnarray}
  This transformation is made to be equiareal.
  So let's compute the determinent of its Jacobian:
  \begin{eqnarray*}
    |\det J| & = & \begin{vmatrix}
                     \frac{\partial X}{\partial \alpha} & \frac{\partial X}{\partial \gamma} \\
                     \frac{\partial Y}{\partial \alpha} & \frac{\partial Y}{\partial \gamma}
                   \end{vmatrix} \\
             & = & \begin{vmatrix}
                     \frac{2\sqrt{2}}{\pi} \cos\gamma & -2\sqrt{2} \frac{\alpha}{\pi} \sin\gamma \\
                                                    0 & \sqrt{2}\cos\gamma
                   \end{vmatrix} \\
             & = & \frac{4}{\pi}\cos^2\gamma
  \end{eqnarray*}
  Thus
  \begin{equation*}
    \mathrm{d}X\mathrm{d}Y = \frac{4}{\pi}\cos^2\gamma \mathrm{d}\alpha\mathrm{d}\gamma
  \end{equation*}
  To be equiareal, we must have (see Eq. \ref{eq:equiareal}):
  \begin{equation*}
    \frac{4}{\pi}\cos^2\gamma \mathrm{d}\alpha\mathrm{d}\gamma \propto \cos\delta\mathrm{d}\alpha\mathrm{d}\delta
  \end{equation*}
  But we know that the surface area of the projection in the plane is equal to $4\pi$, i.e. 
  the same value as for orginial coordinates.
  So we must replace the previous proportionality by a strict equality,
  and we solve integrating on both sides:
  \begin{eqnarray*}
    \frac{4}{\pi}\cos^2\gamma \mathrm{d}\alpha\mathrm{d}\gamma & = & \cos\delta\mathrm{d}\alpha\mathrm{d}\delta \\
    \frac{4}{\pi} \int \frac{\cos(2\gamma) + 1}{2}\mathrm{d}\gamma & = & \int \cos\delta\mathrm{d}\delta \\
    \frac{2}{\pi} (\frac{1}{2}\sin(2\gamma) + \gamma) & = & \sin\delta
  \end{eqnarray*} 
  So we finally find the transcendental equation
  \begin{equation}
    \pi \sin\delta = 2\gamma + \sin(2\gamma)
    \label{eq:mollweide}
  \end{equation}

  \subsubsection{Projection}

    Given the previous section, to compute Mollweide's projection given
    a couple $(\alpha, \delta)$, first solve the previous transcendental
    equation Eq. (\ref{eq:mollweide}) using e.g. the Newton-Raphson method (numerical method).
    So use:
    \begin{equation}
      x_{i+1} = x_i - \frac{f(x_i)}{f'(x_i)}
    \end{equation}
    in which
    \begin{eqnarray*}
      f(x) & = & 2x + \sin(2x) - \pi \sin\delta \\
     f'(x) & = & 2 (1 + \cos(2x)) % = 4\cos^2(x)
    \end{eqnarray*}
    For simplicity, we may use $x'=2x$ and (multiplying both side by a 2)
    use the iteration:
    \begin{equation}
      x'_{i+1} = x'_i - \frac{x'_i + \sin x'_i - \pi \sin\delta}{1 + \cos(x'_i)}
    \end{equation}

    Once $\gamma$ has been obtained numerically, we are able to apply the projection equations:
    \begin{eqnarray}
      X & = & 2\sqrt{2} \frac{\alpha}{\pi} \cos\gamma \label{eq:proj.mol.X}\\
      Y & = &  \sqrt{2}                    \sin\gamma \label{eq:proj.mol.Y}
    \end{eqnarray}

  \subsubsection{Deprojection}

    We multiplying both projection equations Eq. (\ref{eq:proj.mol.X}) and (\ref{eq:proj.mol.Y}),
    leading to
    \begin{equation}
      XY = 2\cos\gamma\sin\gamma\frac{2\alpha}{\pi},
    \end{equation}
    and using Eq. (\ref{eq:sin2x}), we find
    \begin{equation}
      \sin(2\gamma) = XY\frac{\pi}{2\alpha} \label{eq:mol.deproj.sin2g}.
    \end{equation}
    From Eq. (\ref{eq:cos2sin2eq1}) and then using Eq. (\ref{eq:proj.mol.Y}) we find
    \begin{equation}
      \cos\gamma = \sqrt{1-\sin^2\gamma} = \sqrt{1 - \frac{Y^2}{2}}.
    \end{equation}
    Injecting this result in Eq. (\ref{eq:proj.mol.X}), we find
    \begin{equation}
      \alpha = \frac{\pi X}{2\sqrt{2 - Y^2}}.
      \label{eq:mol.deproj.alpha}
    \end{equation}
    Before apllying this equation, we have to check the special value $Y=\pm \sqrt{2}$
    (or in Software $2 - Y^2\le 0$), in which case
    \begin{eqnarray}
       \alpha & = & 0, \\
       \delta & = & \pm\frac{\pi}{2},
    \end{eqnarray}
    the sign of $\delta$ being the same as the sign of $Y$.\\
    From Eq. (\ref{eq:mollweide}):
    \begin{equation}
      \delta = \arcsin \frac{2\gamma + \sin(2\gamma)}{\pi}.
    \end{equation} 
    We have the expression of $\gamma$ in Eq. (\ref{eq:proj.mol.Y})
    \begin{equation}
      \gamma = \arcsin\frac{Y}{\sqrt{2}},
    \end{equation}
    and from Eq. (\ref{eq:mol.deproj.sin2g}) and (\ref{eq:mol.deproj.alpha}) we obtain
    \begin{equation}
      \sin(2\gamma) = Y\sqrt{2 - Y^2}.
    \end{equation}
    We deduce from the three above equations that
    \begin{equation}
      \delta = \arcsin\frac{2\arcsin\frac{Y}{\sqrt{2}} + Y\sqrt{2 - Y^2}}
                           {\pi}.
      \label{eq:mol.deproj.delta}
    \end{equation}
    The deprojection formulae are Eq. (\ref{eq:mol.deproj.alpha}) and (\ref{eq:mol.deproj.delta}).
    

    %\begin{eqnarray}
    %  \alpha & = & \frac{\pi X}{2\sqrt{2}\sqrt{1-\frac{Y^2}{2}}} = \frac{\pi X}{2\sqrt{2-Y^2}} \\
    %  \delta & = & \arcsin\left(\frac{2\asin(\frac{Y}{\sqrt{2}})}{\pi} + \frac{}{\pi} \righ)
    %\end{eqnarray}





  \bibliography{biblio}

\end{document}

